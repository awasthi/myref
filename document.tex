\documentclass{article}



\title{Design \& Analysis of Modified Conditional Data
	Mapping Flip-Flop to }
\author{Soni Singh \and Himani Mittal \and Third Author 
}

%\date{}


\begin{document}
	
	\maketitle
	
	 \begin{abstract}
 	
 	In the history, the major issues of the VLSI designer were area, cost, performance, and reliability; power concern was
 	typically of only lesser
 	importance. But more than the last few years’ power in the circuit is the major difficulty at the present days which
 	is being faced by the very large scale integration industries. The power dissipation in 
 	several circuits is typically take place by the
 	clocking system which includes the clock distribution system and sequential elements (flip flops and latches) in it. The quantity of power
 	dissipation by any clock distribution system and sequential circuit in any chip is as regards of 30\% to 60\% of the overall chip power
 	dissipation by the 
 	circuit. Clock is the most vital signal present in the chip. Clock signals are synchronizing signals which offer timing
 	references 
 	for computation of in the least work in synchronous digital systems. In this paper the power of the sequential circuit is 
 	reduced which in position reduce the on the whole power of the chip. Here dissimilar low power techniques for the lowering static power
 	dissipation are second-hand in the sequential circuit are surveyed. The work analyses the power consumption and propagation delay of flip- flop designs. In Tanner CMOS technology designs are implemented.
 	
	 \end{abstract}
	
\section{Introduction}	
	In the past, the major concerns of the VLSI designer were
	area, performance, cost and reliability. Power consideration
	was mostly of only secondary importance. In recent years,
	however, this has begun to change and, increasingly, power
	is being given comparable weight to area and speed
	considerations. One of the important factors is that excessive
	power Consumption is becoming the limiting factor in integrating
	more transistors on a single chip or on a multiple-chip
	module. 
	
	\[
	\frac{d^2y}{dx^2} + \mu \alpha \frac{dy}{dx}  = 45
		\]
	
	\[ x^{y^{z^2}}  \sum_{i=0}^{\infty} x^2\]
	
	
	Power consumption is dramatically reduced; the
	resulting heat will limit the feasible packing and
	performance of VLSI circuits and systems. Most of the
	current designs are synchronous which implies that flipflops
	and latches are involved in one way or another in the
	data and control paths. One of the challenges of low power
	methodologies for synchronous systems is the power
	consumption of the flip-flops and latches. 
	
	Flip-Flops and latches are the basic elements for storing
	information. One latch or Flip-Flop can store one bit of
	information. The main difference between latches and flipflops
	is that for latches, their outputs are constantly affected
	by their inputs as long as the enable signal is asserted. In
	other words, when they are enabled, their content changes
	immediately when their input change. Flip-flops, on the
	other hand, have their content change only either at the
	rising or falling edge of the enable signal. This enable signal
	is usually the controlling clock signal. After the rising or
	falling edge of the clock, the flip-flop content remains
	constant even. There are basically four main types of latches
	and flip-flops: SR, D, JK, and T. For each type, there are
	also different variations that enhance their operations. Figure
	1(a), (b) illustrates the difference between positive edge
	triggered flip flop and an active high latch. At the same time
	as it can be seen in this figure, likely changes of input can be
	seen at the output of the latch as it is transparent. 
	The presentation of a flip-flop is calculated by three main
	timings and delays: propagation delay (Clock-to-Output),
	setup time and hold time. Setup time and hold time define
	the connection between the clock and input data as shown in
	the Figure 1(c). Setup time and hold time explain the timing
	necessities on the D input of a Flip-Flop with admiration to
	the Clk input. Setup and hold time describe a window of
	time which the D input have to be valid and stable in order
	to guarantee valid data on the Q output. Setup Time (Tsu)
	Setup time is the time that the D input must be valid before
	the Flip-Flop samples. Hold Time (Th) – Hold time is the
	time that D input must be maintained suitable after the FlipFlop
	samples. Propagation Delay (Tpd) – Propagation delay
	is the time that takes to the sampled D input to propagate to
	the Q productivity. 

	
	 There is a broad collection of flip-flops in the journalism \cite{designZhao2011}. Many modern microprocessors selectively employ
	 master-slave and pulsed-triggered flip-flops \cite{EfficientUmadevi2013}.
	 Conventional master-slave single-edge flip-flops, for
	 illustration, transmission gated flip-flop [3], are completed
	 up of two stages, one master and one slave. One more edgetriggered
	 flip-flop is the sense amplifier-based flip-flop
	 (SAFF) [4]. Every one of these hard edged-flip-flops is
	 characterized by means of a positive setup time; cause large
	 D-to-Q delays. On the other hand, pulse-triggered flip-flops
	 reduce the two stages into one stage and are characterized by
	 the soft edge possessions. 95% of all static timing latching
	 on the Itanium 2 processor employs pulsed clocking [5].
	 Pulse triggered flip-flops might be classify into two types,
	 implicit-pulsed and explicit-pulsed, for illustration, the
	 implicit pulse-triggered data-close-to-output flip-flops (ipDCO)
	 [6] and the explicit pulse-triggered data-close-tooutput
	 flip-flops (ep-DCO) [6]. The relationship between
	 Conventional Conditional Data Mapping Flip-flop and
	 Clock Pair Shared D flip flop (CPSFF) here we are
	 examination the working of CDMFF and the conventional D
	 Flip-flop [7]. CDMFFs for low-power and high-performance
	 flip-flops, namely conditional data mapping flip-flops
	 (CDMFFs), which reduce their dynamic power by mapping
	 their inputs to a arrangement that eliminates redundant
	 internal transitions. We nearby two CDMFFs, having
	 differential and single-ended structures, correspondingly,
	 and contrast them to the state-of-the-art flip-flops. It is vital
	 to save power in these flip-flops and latches with no
	 compromising state integrity or performance [8]. High speed
	 parallel counter is scheming to get high operating frequency
	 and to reduce the power consumption, and it can be achieved
	 from side to side a novel pipeline portioning technology [9].
	 The contrast between Conventional Conditional Data
	 Mapping Flip-flop and Clock Pair Shared D flip flop
	 (CPSFF) here we are checking the working of CDMFF and
	 the conventional D Flip-flop.
	 
	 Due to the immense growth in nanometer technology [10], A
	 low power pulse triggered Flip-flop (P-FF) design is done by
	 the pulse generation control logic, an AND function, is
	 uninvolved from the critical path to facilitate a faster
	 discharge operation. A conditional pulse enhancement
	 technique is devised to speed up the discharge the length of
	 the critical path only when needed [11].
\section{Surveyed Low Power Flip Flop Design
	Techniques}
	Power consumption is depends on more than a few factors,
	the total power is sum of dynamic, short circuit and leakage
	power dissipation. Dynamic power indulgence is function of
	frequency, supply voltage, data activity. Based on these
	factors, there are a variety of ways to lower the power
	consumption shown as follows.
	
\subsection{Double Edge Triggering}
    By means of half frequency on
	the clock distribution network will save just about half of the
	power consumption on the clock distribution network. On
	the other hand the flip-flop must be able to be double clock
	edge triggered.
	For example, the clock division shared implicit pulsed flipflop
	(CBS-ip DEFF), is a double edge triggered flip-flop.
	Double clock edge triggering technique reduces the power
	by falling frequency f in equation.

\subsection{Using a low swing voltage} on top of the clock
	distribution network can diminish the clocking power
	consumption since power is a quadratic function of voltage.
	To use low swing clock sharing, the flip-flop should be a
	low swing flip- flop.
	For example, low swing double-edge flip-flop (LSDFF)
	is a low swing flip-flop. In adding up, the level converter
	flip-flop is a natural applicant to be used in low swing
	atmosphere too. For example, CD-LCFF-ip might be
	used as a low swing flip-flop since received signals only
	make nMOS transistors. The low swing technique
	reduces the power utilization by declining voltage in
	equation. 
\section{Conclusion}
We bring to a close this paper by exactness an significant set
of guiding principle which are the curve stone for low power
flip-flop drawing methodology and low power flip-flop
simulation .In general, low power design for combinational
and sequential circuits is an main field and gaining more
significance as time goes by and will continue an vital area
of investigate for a long time. We have accessible a survey
and evaluation of low-power flip-flop circuits. Our new
results enabled us to recognize the power and show tradeoffs
of the flip-flop design. Read Book \cite{StatAmit2018}\cite{NoUma2013}.

\bibliographystyle{plain}
\bibliography{myref}


\end{document}